%% *************************************************************************
%%
%% This is a derivative work of based on the project template:
%% https://github.com/philiplinden/project-definition-doc-template
%%
%% This document uses IEEEtran.cls, the official IEEE LaTeX class
%% for authors of the Institute of Electrical and Electronics Engineers
%% (IEEE) Transactions journals and conferences.
%%
%% *************************************************************************

%% *************************************************************************
% LaTeX REFERENCES
% ----------------
%   Intro to LaTeX: http://www.rpi.edu/dept/arc/docs/latex/latex-intro.pdf
%   Comprehensive LaTeX symbol list: http://tug.ctan.org/info/symbols/comprehensive/symbols-a4.pdf
%% *************************************************************************

% tell \LaTeX what kind of formatting to use
\documentclass[conference]{IEEEtran} % http://www.ctan.org/pkg/ieeetran
\usepackage{graphicx} % enable toolbox for embedding figures and pictures
\usepackage{siunitx} % enable package for easily formatting units
% how to use hyperref: http://www2.washjeff.edu/users/rhigginbottom/latex/resources/lecture09.pdf
\usepackage[T1]{fontenc} % change text encoding to make it more crisp
\usepackage{etoolbox} % enable conditionals for help text
\usepackage{tabularx} % add stretchy cells and autosizes to tables
\usepackage{booktabs} % make beautiful tables!
\usepackage{hyperref} % enable package for cross-referencing figures, sections, references etc.

% set title. choose something as descriptive and precise as possible. Descriptive > sounding cool. remember this!
\title{A project design document template with examples}

\author{
  % List the authors of the design document. The Champion should go first.
  % The \$~\$ markers tell \LaTeX{} to treat the text inside to be treated as a math expression. This way you can use operators like \textcaret{} to place characters as superscripts.
  % Some \LaTeX{} templates handle the author block in different ways. For example, the \href{http://www.worldscientific.com/worldscinet/jai}{Journal of Astronomical Instrumentation} requires the authors' addresses and emails to be included as well.
  % The \textbackslash{}thanks command puts the contents inside those brackets in a footnote at the bottom of the first page. Technically speaking, \textbackslash{}thanks is just a specially formatted footnote.
  % IEEE also has a ``long form'' author block for many authors. Check here for more information:
  % \url{https://tex.stackexchange.com/questions/156523/multiple-authors-with-common-affiliations-in-ieeetran-conference-template}
  % Read here for a more advanced options to modifying footnotes in the author block:  \url{http://tex.stackexchange.com/questions/826/symbols-instead-of-numbers-as-footnote-markers}
  
  %% LONG FORM
  \IEEEauthorblockN{% This block is for author Names.
    Philip~Linden\IEEEauthorrefmark{1},  %the number in the bracket is a reference number to identify this footnote. \LaTeX will figure out what symbol to put there.
  }
  \IEEEauthorblockA{% This block is for the author Affiliations, aka department and university
    Linden Scientific \\ %\\ starts a new line
    Rochester, N.Y. \\
    \IEEEauthorrefmark{1}lindenphilipj@gmail.com
  }

  %% SHORT FORM
  %%   Below, we use the short-form author block and basically hack it to suit our needs.
  % Philip~Linden$^{*\dagger}$%
  %   \thanks{$^{*}$Project Champion}%
  %   \thanks{$^{\dagger}$BS/MEng '17, Mechanical Engineering}

  %%   If there are many authors, consider using symbolic, numeric (aka arabic),  alphabet footnotes or a combination thereof.
  %% the recommended order for symbolic footnotes is
  %%   (1) asterisk        *   *
  %%   (2) dagger          †   \dagger
  %%   (3) double dagger   ‡   \ddagger
  %%   (4) section symbol  §   \S
  %%   et cetera. For higher counts, use 2x symbols (1)-(4) (i.e. (5) two asterisks **). Keep cycling through (1)-(4) using 3x, 4x, and so on.
  %%   Note that these symbol codes work in math mode and text mode.
  %%   There are ways to make LaTeX do this for you, but it is more advanced and not entirely necessary, especially for short author lists. Not worth the hassle, in my opinion.
}

% Initial setup is over, start building the document itself
\begin{document}
\maketitle%
% correct bad hyphenation here, separated by spaces
\hyphenation{explor-ation}

\begin{abstract}
      % The abstract is a brief summary of the design document. Typically it includes the purpose of the design document, key goals or objectives, and justifications.
      % Be sure not to confuse the abstract with the introduction.
      % It is easiest to write the abstract after the rest of the paper has been written.
      % That way you can choose key information from the sections that you've already completed and string them together in the abstract.
      % Consider the abstract to be your elevator pitch to anyone reading this design document.
      % What are they reading?
      % What is the goal?
      % Why is it worth my time?
      % The abstract is what will show up in Google results and other search engines, and what people will read when they are deciding what is worth their time and brain power.
\end{abstract}
% HELPFUL HINTS
% 1. If you get the linter warning ``Command terminated with space.'' when using a \command try placing ``%'' or ``{}'' immediately following the command.
% 2. For proper quotes, begin with `` and close with ''. For single quotes, use '. Double quotes characters copied from Word or Docs (") will show up as weird characters.

% The sections included here are required. Additional sections and subsections may be added as necessary.
\section{Introduction}
\label{sec:introduction}
  % The introduction is a place to give background and context before diving into the subject matter.
  % Establish context for the work you are about to propose and the main ideas of the proposition itself.

\section{Primary Objective}
\label{sec:primary-obj}
  % At the end of the day, whether the project ``succeeds'' or ``fails'' is judged against the objectives it sought to meet.
  % Note that results that contradict expectations/hypotheses are not failures if the scientific \& engineering methods are followed along the way.
  % Sometimes our expectations are wrong and that can be just as successful as getting data we thought we'd see.
  % What matters are what questions you intend to answer.
  % This is the main purpose or main goal the project hopes to achieve.

% % FORMATTING EXAMPLES
% % --------------------
% \autoref{tab:long-example} lists a the relative level of detail expected of the documents written at each stage of a project's life.

% \begin{table*}
% % this table is too wide for the two-column format, so we let it expand across both columns
% % we haven't told LaTeX where to put this so it'll find the best place.
%     \caption{Relative detail expected at each stage of project development.}
%     \centering
%     \begin{tabularx}{\textwidth}{@{}lXcc@{}}
%         % READ THIS!! https://www.inf.ethz.ch/personal/markusp/teaching/guides/guide-tables.pdf
%         %
%         % Normal tables are made using \begin{tabular}, but some extra features are
%         % available to us if we use the tabularx package, including autosized cells
%         % and word-wrap within cells containing lots of text.
%         %
%         % \begin{tabularx}{WIDTH OF TABLE}{ALIGNMENT OPTIONS}
%         % we can use our regular column alignment settings like l, r, and c like normal
%         % tabularx adds the X column alignment option, which autosizes the cell and 
%         % word wraps within the cell. X inherits from l.
%         %
%         \toprule % line on top external edge of table
%         % Separate cells in a row with &, move to the next row with \\
%         Document & Purpose & Contributors & Destination \\
%         \midrule % line separating two internal rows
%         Project Definition Document & To define the goals and requirements of a project. & 2--3 people & Archive \\
%         Project Plans & Specific plans for when work is to be done (Gantt charts) & 2--3 people & Project Repository \\
%         Design Reviews & To review designs before work is started. & 6--8 people & Project Repository \\
%         Test Procedures & Specific instructions and data logs for tests. & 3--4 people & Project Repository \\
%         User Manual & Instructions for future users of project deliverabels. & 3--4 people & Project Repository \\
%         Posters \& Presentations & Materials for sharing projects with the public. & 5--6 people & Project Repository \\
%         Technical Report & Final technical summary of work done and results. & 6 or more & Archive, Conferences \& Journals \\
%         % LaTeX doesn't really like multi-line cell contents. Try to keep the text in each cell concise!
%         \bottomrule
%     \end{tabularx}
% \label{tab:long-example}
% \end{table*}

% \begin{table}[hb!]
%   % the "h" in these brackets tells LaTeX to put the table Here. Try [t] for top and [b] for bottom,
%   % or [hbp] for "here, or if you can't do that put it at the bottom of the page, or if you can't do that put it on its own page.
%   % Here we've also used an "!" to yell at LaTeX to DO THIS OR ELSE!
%   \caption{Notional timeline of Project Milestones.}
%   \centering
%   \begin{tabular}{@{}cll@{}}
%   % the letters here ^^^^ designate the columns.
%   % (l=left align, c=center, r=right align)
%   % the weird @{} thingies tell LaTeX to not have left-right padding between cells
%   % so cells butt up right against the edge
%   \toprule
%   Phase & Task & Duration \\
%   \midrule
%   1 & Review existing designs and materials & 2 weeks or less\\
%   2 & Subsystem development & 6 weeks \\
%     & Order PCB design and/or assembly & 6 weeks \\
%     & Review changes and order materials & 2 weeks or less\\
%     & Testing of individual subsystems & 2 weeks \\
%   3 & System assembly & 1 week  \\
%   4 & System testing & 2 weeks  \\
%   5 & Generate documentation and delivery to SPEX & 1 week  \\
%   \bottomrule
%   \end{tabular}
% \label{tab:short-example}
% \end{table}

\section{Secondary Objectives}
\label{sec:secondary-obj}
  % Secondary Objectives are lower priority or bonus objectives that are significant but not the main focus of the project. This template does not have secondary objectives.

\section{Benefit to (your organization)}
\label{sec:benefit}
  % One of the goals of this template is to provide opportunities for academic and professional growth for its authors,
  % and to challenge them with interesting projects.
  % In this section, explain how the project would benefit the authors as students,
  % space enthusiasts, and young professionals.

  % Below I have used subsections to identify key ideas in this section. These particular subsections are not required as part of the template, but serve as an example of using subsections in a text.

% \subsection{Mindset}
% \label{subsec:mindset}
% Firstly, it gets people in the right mindset for thinking about what is important and what needs to be considered before taking off on a project.
% Publishing a PDD imbues a sense of formality that hopefully makes its way into the level of seriousness and merit that is desirable to pursue.

% \subsection{Traceability}
% \label{subsec:traceability}
% Similarly, a PDD serves to provide the foundation for traceability in requirements and objectives to projects as they grow and change.
% This prevents blockers such as feature creep, rabbit holes, and spun tires, and hopefully prevents good projects from dying by getting too off track.

% \subsection{Accessibility}
% \label{subsec:plug-n-play}
%   % Note below that LaTeX uses weird formatting when it comes to quotation marks.
%   % The style below is correct to display forward quotes `` at the start of the phrase and backquotes '' at the end.

% Having a ``plug-and-play'' template is the first step to learning how to one's own PDD\@.
% It removes a major barrier of starting from scratch, providing example content to which one could refer when creating their own.
% \LaTeX{} may prove to be daunting for some people, but it is arguably better to encourage people to learn LaTeX than to rely on something like Microsoft Word~\cite{lamp94}.

\section{Implementation}
\label{sec:implementation}
  % What path do you anticipate the project to take?

\subsection{Deliverables}
\label{subsec:deliverables}
  % When all is said and done, what will you have to show for it?
  % Examples: Hardware, software, poster, ImagineRIT demo, presentations, technical papers...

\subsection{Milestones}
\label{subsec:milestones}
  % Be as detailed as you can, but it's okay if there are unknowns.
  % At the very least, specify how many semester you expect the project to take until it reaches completion.

\section{Externalities}
  % Things not directly related to the work or outcomes, but related to the project as a whole.
\subsection{Prerequisite Skills}
  % Which skills do team members need to have before work can start (not including skills that will be learned ``on the job'')?

\subsection{Funding Requirements}
  % Estimate costs that would be needed to meet objectives.

\subsection{Faculty Support}
  % Identify faculty that will be involved (or would need to be involved) to meet objectives.
  % Note that if a professor is the Principal Investigator (P.I.) for a project, there still needs to be a student as the Project Champion.

% \subsection{Long-Term Vision}
% \label{sec:vision}

\section*{Acknowledgements}

\bibliographystyle{IEEEtran}
% \bibliography{sample-with-examples}

\onecolumn
\appendices{}
% \section{Project Life Cycle}

\end{document}
